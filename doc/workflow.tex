%%%%%%%%%%%%%%%%%%%%%%%%%%%%%%%%%%%%%%%%%%%%%%%%%%%%%%%%%%%%%%%%%%%%%%%%%%%%%%%
% NAME:             workflow.tex
%
% AUTHOR:           Ethan D. Twardy
%
% DESCRIPTION:      Description of the Workflow I plan to use when developing
%                   this program.
%
% CREATED:          06/06/2018
%
% LAST EDITED:      06/10/2018
%%%
\documentclass[12pt]{article}
\pagestyle{empty} % Prevent style conflicts with `fancy'
\usepackage[margin=1in]{geometry} % Pretty much just to set the margins
\usepackage{fancyhdr} % Header & Footer
\usepackage{setspace} % Spacing.
\usepackage{graphicx} % Images
\usepackage{hyperref}
\hypersetup{
  colorlinks,
  citecolor=black,
  filecolor=black,
  linkcolor=black,
  urlcolor=black
}

% Header
% pdflatex will probably complain about \headheight.
\setlength{\headheight}{28pt}
\pagestyle{fancy}
\fancyhf{}
% Header
\rhead{Ethan D.Twardy}
\lhead{Autoscrum Workflow Definition}

% This may come in handy.
%% \setlength{\emergencystretch}{15pt}
\setlength{\parskip}{10pt}
\setlength{\parindent}{0pt}

\title{Autoscrum Workflow Definition}
\author{Ethan D. Twardy}

\begin{document}
\doublespacing

\maketitle
\pagebreak

\tableofcontents

\section{Introduction}
This document describes in detail the development process which will be used to
effectively realize the Autoscrum application. The creation of the product will
follow a modified Test-Driven Development framework, wherein design
requirements will be set forth, and iteratively the architecture of the product
will be delivered. As each measureable level of the architecture is defined,
tests for the relevant portions will be developed. After tests and requirements
for the product have been fully developed, the tests will be implemented, and
the development of the initial release will be executed. In the next section,
a visual representation of the project flow will be given, and then each
portion of the flow will be described in detail.
\pagebreak

\section{Development Flowchart}
\begin{figure}[hb]
  \includegraphics[width=\textwidth]{assets/designflow.png}
\end{figure}
\pagebreak

\section{Workflow Description}
\subsection{Problem Statement}
In the earliest stage of development a document will be drafted which, at the
highest level will be a precise summary of the issue that the product will
solve, and how the product intends to solve it. This will be an informal but
comprehensive description of the system and its main functionality. This
document will henceforth be referred to as the {\it Problem Statement}.

\subsection{Design Requirements}
A comprehensive set of {\it black box} requirements which describe all
functionality of the system that will be exposed to the user. This document
will be modular, composed of many standalone documents--each only a page or so.
Each one of these documents will define one design requirement which will be
specific and measureable. In total, these design requirements will define all
behavior of the system for any relevant combination of inputs. These design
requirements are not to be confused with implementation details.

\subsection{Design Req. Review}
See the section on \hyperref[sec:review]{Design Review processes}.
\pagebreak

\subsection{System Test Definition}
This is where system level tests are designed to verify that the system meets
every design requirement. These tests should be unambiguous in definition,
verifiable, and there should be a clear and consistent traceability between
the design requirements and each of the system tests. This document will most
likely also be split into many smaller documents.

\subsection{System Test Review}
See the section on \hyperref[sec:review]{Design Review processes}.

\subsection{High-Level Architecture (HLA)}
In this phase, the major components of the implementation will be defined.
Specifically, well-defined interfaces will be created to allow all of the
high-level modules to communicate and perform the duties of the system in a
way that hides the implementation of the modules from each other. This
process is taken from the principles of Object-Oriented design.

\subsection{Integration Test Definition}
Here, tests will be written which stress all functionality of the component
interfaces. Most likely, these tests will take the form of ``unit tests'' which
will stress the high-level modules with a wide range of inputs, to ensure that
no set of inputs will cause the modules to err or break design requirements.

\subsection{Integration Test Review}
See the section on \hyperref[sec:review]{Design Review processes}.

\subsection{Unit Definition}
Here, the structure of the low level modules (here referred to as {\it Units})
are define. Additionally, the behavior and implementation of each of these
units are laid out. Traditionally, a unit is an atomic element of the design.
For example, in a software design, a ``unit'' could be a class or a function.

\subsection{Unit Test Design}
This is the design stage for unit tests. This test design stage follows the
same design workflow as the rest of the test design stages. However, contrary
to the other test categories, these tests will be exclusively implemented in
code.

\subsection{Unit Test Analysis}
See the section on \hyperref[sec:review]{Design Review processes}.

\subsection{Test Implementation \& Execution}
In this section tests will be implemented and the design of the first release
of the final product will be executed, in that order. The reader should note
that design is an iterative process, and it may become necessary to alter one
or more of the requirements or tests during the completion of the release.
The purpose of the lengthy design phase is to minimize this risk.

\subsection{Design Reviews}
\label{sec:review}
The purpose of each of the design review phases is to perform a retrospective
analysis on the two preceding design phases. If there is a problem with the
previous design phase, it may have stemmed from an earlier stage of
development. The cause for the thoroughness is to streamline the process once
actual implementation begins, and to prevent errors from permeating through to
later stages in the design process.

\end{document}

%%%%%%%%%%%%%%%%%%%%%%%%%%%%%%%%%%%%%%%%%%%%%%%%%%%%%%%%%%%%%%%%%%%%%%%%%%%%%%%
